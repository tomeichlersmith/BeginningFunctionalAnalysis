\documentclass[../../Solutions.tex]{subfiles}

\begin{document}

\begin{itemize}
	\item [3.1.1] Suppose $\mathcal{B}$ is countable.
		Then enumerate the elements according to the one-to-one correspondence with $\mathbb{N}$.
		\begin{equation*} \begin{split}
			a_1 & = \{a_{11},a_{12},a_{13},\dots\} \\
			a_2 & = \{a_{21},a_{22},a_{23},\dots\} \\
			a_3 & = \{a_{31},a_{32},a_{33},\dots\}
		\end{split} \end{equation*}
		Define a new sequence in $b \in \mathcal{B}$ as
		\begin{equation*} b_n = \begin{cases}
			1 & \text{if } a_{nn} = 0 \\
			0 & \text{if } a_{nn} = 1
		\end{cases} \end{equation*}
		But then $b$ cannot be represented in the enumeration above.
		Thus $\mathcal{B}$ cannot be countable.
	
	\item  [3.1.2] $\mathcal{B}_T$ is countable because each member $\{a_i\} \in \mathcal{B}_T$ can be identified with a terminating series
		$$ \{a_i\} \to \sum_{i=1}^\infty \frac{a_i}{2^i} $$
		which means each element of $\mathcal{B}_T$ is identified with a unique rational.
		Thus $\mathcal{B}_T$ is in one-to-one correspondence with a subset of $\mathbb{Q}$. \\
		Now, each number $r \in (0,1)$ can be written as a non-terminating binary expansion
		$$ r = \sum_{i=1}^\infty \frac{a_i}{2^i} \quad\text{where } a_i \in \{0,1\} $$
		Then these $\{a_i\}$ represent the members of $\mathcal{B}\setminus\mathcal{B}_T$ (because they are non-terminating).
		Thus $\mathcal{B}\setminus\mathcal{B}_T$ is uncountable which means $\mathcal{B}$ is also uncountable.
	
	\item [3.1.4] Let each member $r \in (0,1]$ be written in a binary expansion $r = 0.a_1a_2a_3\dots$.
		Define $\phi:(0,1]\to(0,1]$ as
		$$ \phi(0.a_1a_2a_3\dots) = 0.a_111a_211a_311\dots $$
		$\phi$ is one-to-one. Suppose $\phi(0.a_1a_2a_3\dots) = \phi(0.b_1b_2b_3\dots)$, then
		$$ 0.a_111a_211a_311\dots = 0.b_111b_211b_311\dots \quad\Longrightarrow\quad a_i = b_i \text{ for all } i \in \mathbb{N} $$
		Thus $0.a_1a_2a_3\dots = 0.b_1b_2b_3\dots$. \\
		Let $r \in \phi((0,1])$. Then $r = 0.r_111r_211r_311\dots$ which means
		$$ \lim_{n\to\infty} \frac{s_n(r)}{n} \to \infty $$
		So $r \in (0,1]\setminus S$.
		Thus $\phi((0,1]) \subseteq (0,1]\setminus S$, which means $(0,1]\setminus S$ is uncountable.
	
\end{itemize}

\end{document}