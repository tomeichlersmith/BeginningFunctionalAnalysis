\documentclass[../../Solutions.tex]{subfiles}

\begin{document}

\begin{itemize}
	\item [2.1.1] Let $E \subseteq (M,d)$ a metric space, and define $E^C = M \setminus E$. \\
		$(\Rightarrow)$ Assume $E$ is open.
		Let $x$ be a limit point of $E^C$, then either $x \in E$ or $x \in E^C$.
		If $x \in E$, then there exists $r > 0$ such that $B_r(x) \subseteq E$ since $E$ is open.
		But $B_r(x) \cap E^C = \emptyset$.
		Thus $x \in E^C$.
		Therefore $E^C$ contains all of its limit points and therefore is closed. \\
		$(\Leftarrow)$ Assume $E^C$ is closed.
		Let $x \in E$.
		Since $E^C$ is closed, there must be an $r > 0$ such that $B_r(x) \cap E^C = \emptyset$.
		(Otherwise, if no such $r$ exists, then $x$ would be a limit point of $E^C$ that is not in $E^C$.)
		Thus $B_r(x) \subseteq \left(E^C\right)^C = E$.
		Therefore $E$ is open.
	
	\item [2.1.2] We will use the equivalence given in Exercise 2.1.1 for parts (b) and (d).
	\begin{enumerate}[(a)]
		\item Assume $\{E_\alpha\}$ is a collection of open sets.
			Let $x \in \bigcup_\alpha E_\alpha$.
			Then $x \in E_\alpha$ for some $\alpha$.
			Thus there exists an $r > 0$ such that $B_r(x) \subseteq E_\alpha \subseteq \bigcup_\alpha E_\alpha$.
			Therefore $\bigcup_\alpha E_\alpha$ is open.
		\item Assume $\{F_\alpha\}$ is a collection of closed sets.
			Then $\{F_\alpha^C\}$ is a collection of open sets.
			Thus (by part (a)) $\bigcup_\alpha F_\alpha^C$ is open, which means its complement is closed.
			Thus $\left(\bigcup_\alpha F_\alpha^C\right)^C = \bigcap_\alpha F_\alpha$ is closed.
		\item Assume $\{E_i\}_{i=1}^N$ is a finite collection of open sets.
			Let $x \in \bigcap_{i=1}^N E_i$.
			Then $x \in E_i$ for all $i$, which means there exists $r_i$ such that $B_{r_i}(x) \subseteq E_i$ for all $i$.
			Set $r = \min{r_i}$.
			Then $B_r(x) \subseteq B_{r_i}(x) \subseteq E_i$ for all $i$.
			Thus $B_r(x) \subseteq \bigcap_{i=1}^N E_i$.
			Therefore $\bigcap_{i=1}^N E_i$ is open.
		\item Assume $\{F_i\}_{i=1}^N$ is a finite collection of closed sets.
			Then $\{F_i^C\}$ is a finite collection of open sets.
			Thus (by part (c)), $\bigcap_{i=1}^N F_i^C$ is open, which means its complement is closed.
			Therefore $\left(\bigcap_{i=1}^N F_i^C\right)^C = \bigcup_{i=1}^N F_i$ is closed.
	\end{enumerate}
	
	\item [2.1.3] The collection $\{(-1,1/n) : n \in \mathbb{N} \}$ is a collection of open sets, but
		$$ \bigcup_{n=1}^\infty \left(-1,\frac{1}{n}\right) = (-1,0] $$
		which is not an open set. \\
		The collection $\{[0,1-1/n] : n \in \mathbb{N} \}$ is a collection of closed sets, but
		$$ \bigcup_{n=1}^\infty \left[0,1-\frac{1}{n}\right] = [0,1) $$
		which is not a closed set.
	
	\item [2.1.4] Let $(M,d)$ be a metric space and let $E \subseteq M$.
		\textit{Note:} We use the claim that $E$ is closed if and only if $E^C$ is open --- this is proven in Exercise 2.1.1. \\
		Assume $E$ is compact and let $x \in E^C$.
		Define $r_e = d(e,x)/2$ for all $e \in E$.
		Define an open cover of $E$ as
		$$ \mathcal{F} = \left\{ B_{r_e}(e) : e \in E \right\} $$
		Since $E$ is compact, $\mathcal{F}$ has a finite sub-cover
		$$ \left\{ B_{r_{e_1}}(e_1), \dots , B_{r_{e_N}}(e_N) \right\} $$
		Let $\rho = \min{r_{e_1}, \dots , r_{e_N}}$.
		Suppose $y \in B_\rho(x) \cap E$. Then $y \in B_{r_{e_i}}(x)$ for some $i$ in the finite cover.
		Then
		$$ d(x,e_i) \leq d(x,y)+d(y,e_i) < \rho + r_{e_i} \leq 2r_{e_i} = d(x,e_i) $$
		which is impossible.
		So there is no $y \in E$ that is in $B_\rho(x)$.
		Thus $B_\rho(x) \subseteq E^C$, which means $E^C$ is open.
		Therefore $E$ is closed.
	
	\item [2.1.5]
	\begin{enumerate}[(a)]
		\item In this space $(0,1)$ is open and not closed.
			\textit{Proof:} Let $x \in (0,1)$.
			Then set $r = \min{x,1-x}$.
			Then $B_r(x) \subseteq (0,1)$.
			Thus $(0,1)$ is open. $0$ is a limit point of $(0,1)$ and $0 \not\in (0,1)$, so $(0,1)$ is not closed.
		\item In this space $(0,1)$ is neither open nor closed.
			\textit{Proof:} The point $(1/2,1) \in (0,1)$ but any ball around it will include points off the x-axis, so $(0,1)$ cannot be open.
			The point $(0,0)$ is a limit point of $(0,1)$ that is not in it, so $(0,1)$ cannot be closed.
	\end{enumerate}
	
	\item [2.1.6] Generally $\interior{A\cup B} \neq \interior{A} \cup \interior{B}$.
		Counter-example: Let $A = (0,1]$ and $B = [1,2)$ in $\R$.
		Then $\interior{A\cup B} = (0,2)$ and $\interior{A}\cup\interior{B} = (0,1)\cup(1,2) \neq (0,2)$. \\
		Contrarily, it is generally true that $\interior{A\cap B} = \interior{A} \cap\interior{B}$. \\
		\textit{Proof:} $(\subseteq)$ Let $x \in \interior{A\cap B}$.
		Then there exists $r>0$ such that $B_r(x) \subseteq A\cap B$.
		This means $B_r(x) \subseteq A$ and $B_r(x) \subseteq B$.
		Thus $x$ is in both interiors, meaning $x \in \interior{A}\cap\interior{B}$. \\
		$(\supseteq)$ Let $x \in \interior{A}\cap\interior{B}$.
		Then there exist $r_1,r_2$ such that $B_{r_1}(x) \subseteq A$ and $B_{r_2}(x) \subseteq B$.
		Set $r = \min{r_1,r_2}$.
		Then $B_r(x) \subseteq A\cap B$.
		Thus $x \in \interior{A\cap B}$. \\
	
	\item [2.1.7] If $M$ has the discrete metric, then the singleton sets are open because
		$$ \{x\} = B_{1/2}(x) $$
		Thus if $E \subseteq M$, it can be written as
		$$ E = \bigcup_{x \in E} \{x\} $$
		which shows that $E$ is the union of open sets, making it open.
		Therefore, all subsets of $M$ are open
		(The collection of open sets in $M$ is $2^M$).
	
	\item [2.1.8]
	\begin{enumerate}[(a)]
		\item The set of limit points of $S=\{\frac{1}{n} : n \in \mathbb{N} \}$ is $\{0\}$.
		\item Because $\mathbb{Q}$ is dense in $\mathbb{R}$, each point of $\mathbb{R}$ is a limit point of $\mathbb{Q}$.
	\end{enumerate}
	
	\item [2.1.9] Define $e_i$ to be the zero sequence with $1$ in the $i$th position.
		Then the set $E = \{ e_i : i \in \mathbb{N}\} \subseteq \ell^\infty$ is closed and bounded, but the open cover $\{B_{1/2}(e_i) : i \in \mathbb{N}\}$ has no finite sub-cover (so $E$ is not compact).
	
	\item [2.1.10] The open cover $\{ (-10 + 1/n , 10] : n \in \mathbb{N} \}$ has no finite sub-cover.
	
	\item [2.1.11] Let $C = \bigcap_{n=1}^\infty C_n$ be the Cantor set where the $C_n$ are the sub-divisions of $[0,1]$ in the usual definition of $C$.
	\begin{enumerate}[(a)]
		\item Since $C \subseteq \R$, we can use the Heine-Borel Theorem to show that it is compact by showing that it is closed and bounded.
			By definition of $C$ (as a subset of $[0,1]$), it is bounded.
			Since $C_n$ is a finite union of closed intervals (for each $n$), $C_n$ is closed for all $n$.
			Since arbitrary intersections of closed sets are closed, $C$ is closed.
			Thus $C$ is closed and bounded and therefore compact.
		\item $\interior{C}=\emptyset$.
			Suppose there is an open interval $I = (a,b)$ (equivalently, an open ball) contained in $C$.
			Then $I \subseteq C_n$ for all $n$.
			The $C_n$ are made up of disjoint intervals of width $(2/3)^n$, so for $I$ to be a subset of all $C_n$, $0 < b-a < (2/3)^n$ for all $n$.
			This is impossible.
			Therefore, there is no open ball contained in $C$.
	\end{enumerate}
	
	\item [2.1.12] Assume $E$ is totally bounded.
		Let $\{ B_{r}(x_i) \}_{i=1}^N$ be the finite cover of $r$-balls.
		Let $x \in E$.
		Define the diameter $D$ as
		$$ D = \max\{ 2r + d(x_i , x_j) : 1 \leq i,j \leq N \} $$
		Claim: $B_D(x) \supseteq E$.
		Let $y \in E$ be given.
		Then $x \in B_r(x_i)$ for some $i$ and $y \in B_r(x_j)$ for some $j$.
		Then
		$$ d(x,y) \leq d(x,x_i) + d(x_i,x_j) + d(x_j,y) < r + d(x_i,x_j) + r \leq D $$
		Thus $y \in B_D(x)$.
		Therefore $B_D(x) \supseteq E$. \\
		Any set with an infinite number of elements would not admit a finite cover of $r$-balls with $r < 1$ under the discrete metric.
	
	\item [2.1.13]
	\begin{enumerate}[(a)]
		\item Define $f_n(x) = x^n$ for all $n$.
			Then $\{f_n\}_{n=1}^\infty$ has no convergent subsequence in $C([a,b])$.
			And $\norm{f_n} = 1$ for all $n$ meaning $f_n \in \conj{B_1(0)}$.
			Thus the closed unit ball is not compact because it is not sequentially compact.
		\item Define $a_n$ as the sequence of zeros with $1$ as the $n$th position.
			Then $\{a_n\}_{n=1}^\infty$ has no convergent subsequence.
			And $\norm{a_n} = 1$ for all $n$ meaning $a_n \in \conj{B_1(0)}$.
			Thus the closed unit ball is not compact because it is not sequentially compact.
		\item %INF DIM TO NOT COMPACT CLOSED UNIT BALL PROOF
	\end{enumerate}
	
	\item [2.1.14] Assume $A \subseteq M$ is compact and $U \subseteq M$ is open.
		Let $\mathcal{E} = \{E_\alpha\}$ be an open cover of $A \setminus U$.
		Then $\{U\} \cup \{E_\alpha\}$ is an open cover of $A$.
		This open cover has a finite sub-cover $\mathcal{F}$ because $A$ is compact.
		Then $\mathcal{F} \setminus \{U\}$ still covers $A \setminus U$, and is still finite.
		Thus $\mathcal{E}$ has a finite subcover, making $A \setminus U$ compact.
		
	\item [2.1.15]
	\begin{enumerate}[(a)]
		\item Let $f:M_1 \to M_2$. \\
			$(\Rightarrow)$ Let $U \subseteq M_2$ be open and let $f$ be continuous.
			Let $y \in M_1$ such that $f(y) \in U$.
			Since $U$ is open, there exists $\epsilon > 0$ such that $B_\epsilon(f(y)) \subseteq U$.
			Since $f$ is continuous, there exists $\delta > 0$ such that if $x \in B_\delta(y)$ then $f(x) \in B_\epsilon(f(y)) \subseteq U$.
			Thus $B_\delta(y) \subseteq f^{-1}(U)$ which means $f^{-1}(U)$ is open. \\
			$(\Leftarrow)$ If $U \subseteq M_2$ is open, then $f^{-1}(U) \subseteq M_1$ is open.
			Let $\epsilon > 0$ be given.
			Then $B_\epsilon(f(x)) \subseteq M_2$ is open, which means $f^{-1}(B_\epsilon(f(x))) \subseteq M_1$ is open.
			Thus there exists a $\delta > 0$ such that $B_\delta(x) \subseteq f^{-1}(B_\epsilon(f(x)))$.
			Therefore $f$ is continuous.
		\item Let $f:M \to \R$ be continuous and let $A \subseteq M$.
			Assume $f(A)$ is unbounded.
			Define $\{y_i\}_{i=1}^\infty \subseteq f(A)$ to be a sequence that diverges to infinity.
			Then $\{x_i:f(x_i) = y_i\}_{i=1}^\infty \subseteq A$ is a sequence that has no convergent subsequence because otherwise $\{y_i\}_{i=1}^\infty$ wouldn't diverge.
			Thus $A$ is not sequentially compact and therefore not compact. \\
			Equivalently, if $A$ is compact, then $f(A)$ is bounded.
	\end{enumerate}
	
	\item [2.2.1] Following the outline given in Example 2, we assume that $\mathcal{P}([a,b])$ is dense in $C([a,b])$.
	\begin{enumerate}[(i)]
		\item Define the function $\phi : \mathcal{Q}([a,b]) \to \bigcup_{n=1}^\infty \mathbb{Q}^n$ by
			$$ \phi(a_0+a_1x+a_2x^2+\cdots + a_nx^n) = (a_0,a_1,a_2,\dots,a_n) $$
			$\phi$ is a clear one-to-one correspondence.
			Thus $\mathcal{Q}([a,b])$ has the same cardinality of $\bigcup_{n=1}^\infty \mathbb{Q}^n$ which is a countable union of countable sets.
			Thus $\mathcal{Q}([a,b])$ is countable.
		\item 
	\end{enumerate}
	
\end{itemize}

\end{document}