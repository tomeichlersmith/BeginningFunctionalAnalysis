\documentclass[../../Solutions.tex]{subfiles}
 
\begin{document}

\begin{itemize}
	\item [A.1]
		\begin{enumerate}[(a)]
			\item Multiplying the numerator and denominator by the complex conjugate produces the necessary simplification.
			$$ \frac{2-5i}{1+i}\frac{1-i}{1-i} = \frac{2-5i-2i-5}{1+1} = -\frac{3}{2}-\frac{7}{2}i $$
			
			\item We are able to show this equality simply by multiplying by $1$.
			$$ \frac{1}{i} = \frac{1}{i}\frac{i}{i} = \frac{i}{i^2} = \frac{i}{-1} = -i $$
		\end{enumerate}
		
	\item [A.2] For this exercise, we write $w = a+ib$ and $z = c+id$.
	The first four of these will be done by direct calculation from the definitions.
	%This means $a = A\cos(\theta)$, $b = A\sin(\theta)$, $c = C\cos(\phi)$, and $d = C\sin(\phi)$.
		\begin{enumerate}[(a)]
			\item $\conj{(w+z)} = \conj{a+ib+c+id} = \conj{a+c+i(b+d)} = a+c-i(b+d) = a-ib+c-id = \conj{a+ib}+\conj{c+id} = \conj{w}+\conj{z}$
			
			\item $\conj{w \cdot z} = \conj{(a+ib)(c+id)} = \conj{ac-bd+i(ad+bc)} = ac-bd-i(ad+bc) = (a-ib)(c-id) = (\conj{a+ib})(\conj{c+id}) = \conj{w} \cdot \conj{z}$
			
			\item Using part (d) and part (b). \\
			$|wz| = \sqrt{wz \conj{wz}} = \sqrt{w \cdot z \cdot \conj{w} \cdot \conj{z}} = \sqrt{w\conj{w}z\conj{z}} = \sqrt{w\conj{w}}\sqrt{z\conj{z}} = |w||z|$
			
			\item $z\conj{z} = (c+id)(c-id) = c^2+d^2 = \sqrt{c^2+d^2}^2 = |c+id|^2 = |z|^2$
			
			\item This inequality is an equality for $n=1$.
			We will prove $n \geq 2$ by induction.
			The statement holds for $n=2$.
			\begin{equation*} \begin{split}
				|z_1+z_2|^2 & = (z_1+z_2)\conj{(z_1+z_2)} \\
							& = (z_1+z_2)(\conj{z_1}+\conj{z_2}) \\
							& = z_1\conj{z_1} + z_2\conj{z_2}+z_1\conj{z_2}+z_2\conj{z_1} \\
							& \leq z_1\conj{z_1} + z_2\conj{z_2} + 2\sqrt{z_1\conj{z_1}z_2\conj{z_2}} \\
							& = |z_1|^2 + |z_2|^2 + 2|z_1||z_2| \\
							& = (|z_1|+|z_2|)^2
			\end{split} \end{equation*}
			Thus $|z_1+z_2| \leq |z_1|+|z_2|$.
			Now assume $N > 2$ and that the statement holds for all $n < N$.
			Since addition of complex numbers is associative (and using the base case above), we can conclude.
			\begin{equation*} \begin{split}
				|z_1 + z_2 + \cdots +z_N| & \leq |z_1+z_2+ \cdots +z_{N-1}|+|z_N| \\
										  & \leq |z_1|+|z_2|+\cdots+|z_{N-1}|+|z_N|
			\end{split} \end{equation*}
			Therefore, by the strong principle of mathematical induction,
			$$ |z_1+z_2+\cdots+z_n|\leq|z_1|+|z_2|+\cdots+|z_n| $$
			is true for all $n$.
			
			\item Since $|z|^2 = x^2+y^2$, we can conclude that $|x| \leq |z|$ and $|y| \leq |z|$. Let $m = \max(|x|,|y|)$. Then $m \leq |z|$ (because both $|x|$ and $|y|$ are) and
			$$ |z|^2 = |x|^2+|y|^2 \leq m^2+m^2 = 2m^2 \Rightarrow |z| \leq \sqrt{2}m $$
			Therefore, $m \leq |z| \leq \sqrt{2}m$.
			
			\item This can be done by direct calculation.
			$$ \frac{z+\conj{z}}{2} = \frac{a+ib+a-ib}{2} = \frac{2a}{2} = a = \Re(z) $$
			$$ \frac{z-\conj{z}}{2i} = \frac{a+ib-a-ib}{2i} = \frac{2ib}{2i} = b = \Im(z) $$	
		\end{enumerate}
	
	\item [A.3] 
		\begin{enumerate}[(a)]
			\item Here we set $w = a+ib$ and $z = c+id$ and use Euler's formula for $e^{i\theta}$.
			Then we can directly calculate the desired equality.
			\begin{equation*} \begin{split}
				e^{w+z} & = e^{a+c+i(b+d)} \\
						& = e^{a+c}(\cos(b+d)+i\sin(b+d)) \\
						& = e^ae^c(\cos(b)\cos(d)-\sin(b)\sin(d)+i(\cos(b)\sin(d)+\sin(b)\cos(d))) \\
						& = e^ae^c(\cos(b)+i\sin(b))(\cos(d)+i\sin(d)) \\
						& = e^a(\cos(b)+i\sin(b))e^c(\cos(d)+i\sin(d)) \\
						& = e^{a+ib}e^{c+id} \\
						& = e^we^z \\
			\end{split} \end{equation*}
			Thus $e^{w+z}=e^we^z$ for any complex numbers $w$ and $z$.
			
			\item We can prove this by contradiction.
			Assume there exists a complex number $z = a+ib$ such that $e^z=0$.
			Then
			$$ e^z = e^a(\cos(b)+i\sin(b)) = 0 $$
			Since $a$ is a real number, $e^a \neq 0$, so
			$$ \cos(b)+i\sin(b) = 0 \Rightarrow \cos(b) = 0 \text{ and } \sin(b) = 0 $$
			But there does not exist a real number $b$ such that both $\cos(b)$ and $\sin(b)$ are equal to zero.
			Therefore, $e^z$ cannot equal zero.
			
			\item Using Euler's formula, we can prove this with direction calculation.
			\begin{equation*}
			|e^{i\theta}| = |\cos{\theta}+i\sin{\theta}| = \sqrt{ \cos^2(\theta)+\sin^2(\theta) } = \sqrt{1} = 1
			\end{equation*}
			Thus $|e^{i\theta}| = 1$.
			
			\item Using Euler's formula and exponent rules, we can prove this statement for any real number $n$ by direct calculation.
			$$ (\cos(\theta)+i\sin(\theta))^n = (e^{i\theta})^n = e^{in\theta} = \cos(n\theta)+i\sin(n\theta) $$
			
			\item $e^z = e^w$ does not imply that $z = w$ (like it does for real numbers).
			Counter-example: Let $z = a+ib$ and $w = a+i(b+2\pi)$.
			Clearly, $z \neq w$, but
			$$ e^w = e^a(\cos(b+2\pi)+i\sin(b+2\pi)) = e^a(\cos(b)+i\sin(b)) = e^z $$
			Therefore, the implication does not hold.
		\end{enumerate}
	
	\item [A.4] 
		\begin{equation*} \begin{split}
			\frac{(\sqrt{3}+i)^6}{(1-i)^{10}} = \frac{(2e^{i\pi/6})^6}{(\sqrt{2}e^{i\pi/4})^{10}}
				 = \frac{2^6e^{i\pi}}{2^5e^{i5\pi/2}} = 2e^{-i3\pi/2} = 2e^{i\pi/2} = 0+2i
		\end{split} \end{equation*}
		So the real part is $0$ and the imaginary part is $2$.
		
	\item [A.5] This equality is not true.
	Counterexample: Let $z = w = i$, then
	$$ \Re(z \cdot w) = \Re(-1) = -1 $$
	But
	$$ \Re(z) \cdot \Re(w) = 0 \cdot 0 = 0 \neq -1 $$
	
	\item [A.6] An ordered field must satisfy two properties for all elements $a$, $b$, and $c$ in the field.
	(In addition to the axioms of a linear order and a field.)
	\begin{enumerate}
		\item If $a \leq b$ then $a+c \leq b+c$.
		\item If $0 \leq a$ and $0 \leq b$, then $0 \leq ab$.
	\end{enumerate}
	The second property leads directly to all squares being positive.
	More specifically, if $a$ is a member of an ordered field, then $a < 0$ or $a \geq 0$.
	If $a \geq 0$, then it follows directly from the second property that $a^2 \geq 0$.
	If $a < 0$, then $-a > 0$ and $(-a)^2 = (-1)^2a^2 = a^2 > 0$.
	Thus $a^2 \geq 0$ for any $a$ in the ordered field.
	But the complex numbers have $i$ as an element and $i^2 = -1 < 0$, so the complex numbers cannot be ordered to give an ordered field.
\end{itemize}

\end{document}